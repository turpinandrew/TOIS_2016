\documentclass[a4paper]{article}
\usepackage{xcolor}
\usepackage[top=20mm, bottom=20mm, left=20mm, right=20mm]{geometry}

\newcommand{\comment}[1]{\vspace{1em} \textcolor{red}{{#1}} \vspace{1em}}

\begin{document}
\vspace{1em}
\vspace{1em}

\noindent 21 September 2016
\vspace{1em}

\noindent Dear Maarten,
\vspace{1em}

\noindent Please find herein our response to reviewer's comments 
for our paper.

\vspace{1em}

\noindent We thank the reviewers for their constructive feedback and suggestions
for improvements.
Our responses to particular comments are outlined below.
Changes in the manuscript have been highlighted in red font to make
them easy to spot.

\vspace{1em}

\noindent Best Regards,

\vspace{1em}
\noindent Andrew Turpin
\vspace{1em}
\vspace{1em}

\hrule
\vspace{1em}
\vspace{1em}




\section{Reviewer 1}

In the footer on the first page, please add a concise statement of
the paper's addition after ``This paper is a revised and extended
version of [Turpin et al. 2015]; it adds XXX." By the way, you
should use [Author year] when you refer to a publication and Author
[year] when you refer to the author (citep{xxx} and citet{xxx},
respectively).

\comment{The citation style has been fixed, thank you. A detailed list of changes has
been added at the end of  Section 2.2, in line with the next
item.}

In the related work section, please make precise how the TOIS
submission differs from the earlier SIGIR 2015 paper.

\comment{A paragraph has been added at the end of Section 2.2 to do this.}

RQ2 is an important additional ingredient of this extended paper.
Section 6 is devoted to answering this question. Please explicitly
announce that that is what you are going to do. And please provide
answers to the question at the end of section 6. We are now left
to discover for ourselves what (if any) answer you were able to
produce for RQ2.

\comment{We have added Section 6.4 which summarises the section.}



\section{Reviewer 2}

I like the paper and the problem that it addresses: magnitude
estimation for information retrieval and the use of crowd sourcing
for this purpose. The authors claim that their findings are direct
implications for IR evaluation. This is probably true, but I think
the authors can do a better job of articulating them. I recommend
that each result section 5-8 explicitly (1) starts with one of the
four research questions, (2) answers it, (3) explains what the
ramifications of the answers are for IR evaluation.

\comment{
This has been done for Section 6 (RQ2) by adding Section 6.4, 
Section~7 (RQ3) by adding Section~7.4, and
Section~8 (RQ4) by adding the final paragraph.
We feel RQ1 is reasonably covered in Section~5.
}

Some more specific comments:

Section 2.3: good addition. You could consider adding a brief
discussion on the value of crowd worker disagreement. There is work
by Lora Aroyo and Chris Welty that you should probably relate to
in section 2.3

\comment {Thanks for pointing this out, we have added a paragraph in Section 2.3.}

Section 4: This is a better way of organizing the material than in
the SIGIR version. Collect and report on the data first. And only
interpret and conclude later. Moving bits from section 3 here (on
crowd judging and score normalization) makes good sense. Figures 1
and 2 are new. And so is section 4.2; adding a sentence to 4.2 with
the upshot of the section would be useful. The expansion of 4.3
(formerly 3.3) is useful.

\comment {We have added the sentence at the end of 4.2}

The new section 5 coincides more  or less with section 4 in the
SIGIR paper. Is ``We therefore next investigate judge agreement"
still a useful  way of ending this section, given the inclusion of
a new research question?

\comment {We have deleted the offending sentence at the end of Section 5.}

Section 6:  ranges of scores for each worker involved was very wide
$\rightarrow$ RANGE of scores. 

\comment {Fixed, thanks.}

Figure 7 has no legends (neither x nor y).

\comment {Curious: there are axis labels, and the symbols are explained
in the caption. Perhaps a printing error on the reviewer's side?} 

What are the practical implications of the statement at the very
end of section 6, about the number of required workers? And what
does it say about setups where a smaller number of workers has been
used?

\comment {We have added Section 6.4 to draw this section together.}

Section 7 coincides with the old section 5. 
Figs 10, 11, 12 need legends. 

\comment {Again, this seems fine on our copies.}

It would be good to explain the differences in absolute
numbers between Table IV in the TOIS submission and Table 2 in the
SIGIR paper. And why these differences do not matter.

\comment {We have added two sentences at the end of the first paragraph in Section~7.2 
explaining the difference. On closer inspection, we also noticed that the new methodology did make a
difference, and have altered the text in the final paragraph of Section~7.3 accordingly.}

Section 8: I am not sure I am happy with the uncertain ending of
this section. What does the disagreement with Kanoulas and Aslam's
earlier work mean for our ``ideal" evaluation setup?

\comment{We have added a summary paragraph to the end of Section~8.1.}

Section 9: the addition of a ``limitations" section is an excellent
idea.

Appendix A: I like this addition. A screen dump of the instruction
interface would have been helpful.

\comment{The exact instructions that appeared on the screen are already
given in the appendix, and a screen dump would just format them in an
HTML/www style, rather than text.
As such, we don't think that an extra screen dump of the instructions
would be informative.
% (In any case, it unfortunately turns out that CrowdFlower has undergone
% an `upgrade' since we ran the experiments, so, sadly, the exact screen
% shot is no longer available anyway.)
} 

\end{document}
