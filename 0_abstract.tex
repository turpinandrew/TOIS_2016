
\begin{abstract}

Magnitude estimation is a psychophysical scaling technique for the
measurement of sensation, where observers assign numbers to stimuli in
response to their perceived intensity.
We investigate the use of magnitude estimation for judging the
relevance of documents for information retrieval
evaluation, carrying out a large-scale user study across 18 TREC topics
and collecting
over 50,000 magnitude estimation judgments using crowdsourcing.
%
Our analysis shows that magnitude estimation judgments 
\textcolor{red}{ can be
reliably collected using crowdsourcing,  are competitive in terms of
assessor cost, and}, on average, they are
rank-aligned with ordinal judgments made by expert relevance assessors. 

%An advantage of magnitude estimation is that users can chose their own
%scale for judgments, allowing deeper investigations of user perceptions
%than when categorical scales are used.

We explore the application of magnitude estimation for IR evaluation,
calibrating two gain-based effectiveness metrics, nDCG and ERR,
directly from user-reported perceptions of relevance.
A comparison of TREC system effectiveness rankings based on binary,
ordinal, and magnitude estimation relevance shows substantial
variation; in particular, the top systems ranked using magnitude
estimation and ordinal judgments differ substantially.
Analysis of the magnitude estimation scores shows that this effect is
due in part to varying perceptions of relevance: different users have
different perceptions of the impact of relative differences in document
relevance.
%We further use magnitude estimation to investigate users' gain
%profiles, comparing the currently assumed linear and exponential models
%of gain with actual user-reported relevance perceptions.
%This indicates that the currently used exponential gain profiles in
%nDCG and ERR are mismatched with an average user, but perhaps more
%importantly that individual perceptions are highly variable.
These results have direct implications for IR evaluation, suggesting
that current assumptions about a single view of relevance being
sufficient to represent a population of users are unlikely to hold.
%Finally, we demonstrate that magnitude estimation judgments can be
%reliably collected using crowdsourcing, and are competitive in terms of
%assessor cost. 

\end{abstract}

% Local Variables:
% TeX-master: "ME-TOIS.tex"
% End:
