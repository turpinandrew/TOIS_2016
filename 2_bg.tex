
\section{Background}
\label{sec-background}

\subsection{Relevance} 
%\myparagraph{Relevance:}
Relevance is an important concept in information science and
information retrieval~\cite{Sar07}.
IR system effectiveness is most often measured with reference to a test
collection, consisting of a set of search topics, a static set of
documents over which to search, and human-generated assessments that
indicate, for an answer document returned by a search system in
response to a topic, whether the document is relevant~\citep{VooHar05}.
Typically, these assessments are made based on ``topical'' relevance,
that is, a judgment of whether the document contains any information
that is ``about'' the material that the search topic is asking for.
Other aspects of relevance, such as novelty or contextual factors, are
not considered~\cite{Sar07}.
Based on the relevance judgments, each ranked answer list returned by
a system is aggregated into a number that reflects the system's
performance, using a chosen effectiveness metric.
The metrics themselves instantiate either implicit or explicit assumptions
about searcher behavior, for example that a relevant document that is
returned by a system lower down a ranked list is of less value
than if the same document had been returned earlier.

Historically, relevance judgments were often made using a 
\emph{binary} scale, where a document is classed as either being
relevant to the search topic or not, and many effectiveness metrics use
this notion of relevance, including 
\emph{precision}, \emph{recall}, 
\emph{mean average precision} (MAP), and \emph{precision at a specified cutoff}~\cite{VooHar05}.
More recently, based on the observation that searchers can generally
distinguish between more than two levels of relevance, evaluation
metrics that incorporate \emph{multi-level} relevance have been
proposed~\cite{JarKek02}.
Here, relevance is typically measured on an ordinal scale; metrics that
include this more fine-grained notion of relevance include 
\emph{normalized discounted cumulative gain} (nDCG)~\cite{JarKek02},
\emph{expected reciprocal rank} (ERR)~\cite{ChaMet09}, and the
\emph{Q-measure}~\cite{Sakai07}.
Some examples of previous choices for the number of levels in ordinal
relevance scales include 3 (TREC Terabyte Track~\cite{ClaCra04}), 4
(TREC newswire re-assessments by~\citet{Sormunen:2002}), and 6 (TREC
Web Track~\cite{ColBen14}).
\citet{TanSha99} studied the self-reported confidence of psychology
students when assessing the relevance of bibliographic records on
ordinal scales from 2 to 11 levels, and observed that for this specific
task, confidence is maximized with a 7-point scale.
However, the issue is far from settled: in a broader survey of the optimal number of levels in an ordinal
response scale, \citet{Cox80} concludes that there is no single number
of response alternatives that is appropriate for all situations.

\subsection{Magnitude Estimation} 

%\myparagraph{Magnitude estimation:}
Magnitude estimation (ME) is a psychophysical technique for the construction
of measurement scales for the intensity of sensations.
An observer is asked to assign numbers to a series of presented
stimuli.
The first number can be any value that seems appropriate, with
successive numbers then being assigned so that their relative
differences reflect the observer's subjective perceptions of
differences in stimulus intensity~\citep{EhrEhr99}.
A key advantage of using magnitude estimation is that the responses are
on a ratio scale of measurement~\citep{Ges97}, meaning that all mathematical operations
may be applied to such data, and parametric statistical analysis
can be carried out.
In contrast, for ordinal (or ranked category) scales certain operations
are not defined; for example, the median can be used as a measure of
central tendency for ordinal data, but the mean is not meaningful since
the distance between the ranked categories is not defined~\citep{She07}.

Proposed by Stanley Stevens in the 1950s, magnitude estimation
has a long history, and is the most widely-used psychophysical
ratio scaling technique~\cite{Ges97}.
Initially developed to measure perceptions of physical stimuli, such as
the brightness of a light or the loudness of a sound, magnitude
estimation has also been successfully applied to a wide range of
non-physical stimuli in the social sciences (including occupational
preferences, political attitudes, the pleasantness of odors, and the
appropriateness of punishments for crimes~\citep{Ste66}), in medical
applications (such as levels of pain, severity of mental disorders, and
emotional stress from life events~\citep{Ges97}), in user experience
research (for example, as a measure of usability in HCI~\citep{McG03},
and for health-care applications~\citep{Joh10}), and in linguistics
(including judging the grammaticality of sentences~\citep{BarRob96}).

In information retrieval research, 
\citet{Eis88} investigated magnitude estimation in the context of
judging the relevance of document citations from a library database
(including fields such as author, title, keywords and abstract), and
concluded that participants are able to effectively use magnitude
estimation in such a scenario.
A related technique was used by 
\citet{SpiGre01}, where
participants in a user study were required to fill in a worksheet with
information about the relevance of resources that were retrieved from a
library database for personal research projects; this included
indicating the level of relevance on a 4-level ordinal scale, providing
feedback about other levels of relevance such as utility and
motivation, and marking the level of relevance on a 77mm line.
The line was then decoded into numbers at a 1mm resolution
so was in effect a 78-level ordinal scale.

To the best of our knowledge, the only previous investigation of
magnitude estimation for document-level relevance judging in the
context of IR evaluation was our own small pilot study~\cite{SchMad14,MadMiz15}.
That study indicated that 
magnitude estimation could be useful for
measuring document-level relevance, but used only 
3 information need statements and 33 documents.
In this paper we carry out a much larger-scale study, across 18 TREC
topics and 4269 documents.
Unlike previous studies, 
our work also explores the application of this scaling approach to the
direct calibration of effectiveness metrics for IR system evaluation,
and the relationship between the magnitude relevance space and the
concept of gain.

This paper is a revised and extended version of
research that appeared at SIGIR 2015~\cite{ME-SIGIR15}.
In particular, this version adds: a new research question of whether
crowdsourcing is a viable method to gather robust magnitude estimation
values (RQ2); related background on crowdsourcing; more information on
the ME score distributions reported in Section~\ref{sec:descr-stat},
including Figures~\ref{fig:ME-raw-scores} and~\ref{fig:ME-norm-scores}
and their discussion; a new Section~\ref{sec:cs} including
Figure~\ref{fig:judgeVariability} and re-computation of
Figure~\ref{fig:workersQuality}; new material in Section~\ref{sec:rq2},
including re-computation of Figures~\ref{fig:runs-rerank-median3best}
and~\ref{fig:runs-rerank-median3best-fake}; a revised second half of
Section~\ref{sec-rq3}; an Appendix giving exact participant
instructions; and additional references.


\subsection{Crowdsourcing}
\label{sec:crowdsourcing}

The third ingredient of our work is crowdsourcing. 
namely the outsourcing of a task usually performed by experts to a
crowd by means of an open call. 
Although criticized by some \cite{keen2008}, crowdsourcing has been
shown to be effective for several tasks, at least under some
conditions, and
% blog posts? 
% Iperiotis?,
it has been (and still is) much used as a method to collect relevance
judgments. 

For example \citet{Alonso:2012} compared relevance judgments gathered
by the crowd vs.\ more experienced TREC assessors, finding comparable
accuracy. 
Others have attempted to measure relevance in various ways, including
graded judgments \cite{mccreadie:2011}, preference-based judgments
\cite{Anderton2012}, and multidimensional ones \cite{Zhang2014}.
Collecting relevance labels through crowdsourcing has also been used
in practice, for example in the TREC blog track \cite{mccreadie:2011},
or in the judgment task of the TREC Crowdsourcing Track
\cite{Smucker2014}. 
Still in the IR field, it has been shown that crowdsourcing can be
effective not only to collect relevance judgments, but also for more
complex tasks \cite{Zuccon:2013}.


Of course, quality is an issue: workers might be too inexperienced for
some tasks, or even behave maliciously to finish the task without any
accuracy, perhaps even in a random way, just to gain money. 
Several papers address this problem.
For example, \citet{Clough2013} show that even if workers disagree,
system rankings might not change. 
\citet{Kazai2013} study how several factors (e.g., pay, effort, worker
qualifications, motivations, and interest) can affect the quality of
the relevance judgments collected.

Appropriate aggregation methods can be used as countermeasures to low
quality work. 
\citet{Hosseini:2012} and \citet{Jung2011} compare the classical
aggregation method of majority voting with more sophisticated
approaches that take into account a worker's expertise (based on
expectation maximization and outlier detection, respectively), finding
the latter to be more effective. 

However, it is important to note that disagreement does
not necessarily imply low worker quality.
As pointed out for example by \citet{aroyo2015truth}, disagreement
might depend on the false assumption of a ``unique truth'', it might
carry useful information that can be exploited, and attempts to
eliminate disagreement, for example by detailed instructions, might not
improve quality and simply lead to information loss.

In this work, we use crowdsourcing to gather magnitude estimation
relevance assessments at scale.
Apart from our own pilot studies \citep{SchMad14,MadMiz15} we
are not aware of any studies that uses crowdsourcing to gather magnitudes,
let alone using crowdsourcing to gather relevance judgments expressed
by magnitude estimation.
Since magnitude estimation is a process that is unlikely to be familiar
to the average crowd worker, analyzing the suitability of the crowd for
completing such tasks is novel.

% Local Variables:
% TeX-master: "ME-TOIS.tex"
% End:


