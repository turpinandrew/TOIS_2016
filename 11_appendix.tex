\section*{Appendix A: Instructions Shown to Participants}
\label{a-instructions}

\subsection*{Introduction}

In this task, you will be asked to make Magnitude Estimation
assessments. Magnitude Estimation is a technique whereby you assign
numbers to indicate your perception of the strength of an effect. It is
described in more detail below.

Please read these instructions carefully before deciding whether to
complete the task. Note that there are some checks throughout the task,
and if you do not perform these correctly you will not be able to
terminate and get paid. The data from this task is being gathered for
research purposes.  No personally identifying information is recorded.
Participation is entirely voluntary, and you are free to discontinue at
any point.


\subsection*{Task Steps}

\noindent \emph{1. Magnitude Estimation of line lengths}

As a warm-up task, to familiarize you with Magnitude Estimation, you
will be shown a sequence of three lines, one at a time. 

\begin{enumerate}
\item[1.a] For each line, enter a number into the text box below the displayed
line. This number should reflect your perception of the length of the
line.  You may use any numbers that seem appropriate to you -- whole
numbers, fractions, or decimals. However, you may not use negative
numbers, or zero. 

\item[1.b] For each subsequent line, enter a number that reflects your
perception of its length, relative to the previous line. For example,
if you feel that the current line is twice as long as the previous, then 
you should assign a number that is twice as large as the number you used
previously.
\end{enumerate}

Don't worry about running out of numbers -- there will always be a
larger number than the largest you use, and a smaller number than the
smallest you use.

Note: the Magnitude Estimation scores are *not* intended to be an
estimate of the length in any particular measurement units, such as
centimetres.

~\\

\noindent \emph{2. Magnitude Estimation of document relevance}

In the main part of this task, you will be asked to assign Magnitude
Estimation scores to indicate your perception of document
relevance. 

\begin{enumerate}
\item[2.a] First, a statement that expresses a need for information will be
displayed at the top of the screen.  Please read the statement
carefully.  You will be asked a question about the statement, to test
your understanding.

\item[2.b] You will then be asked to rate 8 documents that have been returned
by a search system, in response to the information need statement.
\end{enumerate}


Your task is to indicate how RELEVANT these documents appear to you, in
relation to the information need. 

As a preliminary exercise, can you imagine a document which would be
highly relevant to the information statement? Can you imagine a document
that you would judge to be low in relevance? Can you imagine a document
that you would judge to be medium in relevance? 

Now do the same for numbers. Imagine of a large number. A small
number. A medium number.

As indicated above, you will be shown 8 documents, one at a time. Your
task will be to assign a number to every document in such a way that
your impression of how large the number is matches your judgment of how
relevant the document is. 

Write the number for each document in the box under the document
description.  

\begin{itemize}
\item You may use any numbers that seem appropriate to you -- whole numbers,
  fractions, or decimals. However, you may not use negative numbers, or
  zero. 

\item Don't worry about running out of numbers -- there will always be a
  larger number than the largest you use, and a smaller number than the
  smallest you use.

\item Try to judge each document in relation to the previous one. For
  example, if the current document seems half as relevant as the
  previous one, then assign a score that is half of your previously
  assigned score. 

\item You are requested to indicate your best judgment of relevance of a
  document at the time it is presented to you, one document at a
  time. However, if you wish to correct a mistake, then you can use
  the back button to revisit a previous judgement.

\item The documents are not presented in any particular order. You might see
  many good documents, many bad documents, or any combination. Try not to
  anticipate, and simply rate each document after reading it.

\item To judge each document, you will need to read it completely.  A
  document's relevance (or nonrelevance) might depend on a small part of
  the document. Don't try to guess, as there are some cross-checks, as
  indicated above.
\end{itemize}


% Local Variables:
% TeX-master: "ME-TOIS.tex"
% End: